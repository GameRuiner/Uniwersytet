\documentclass[12pt]{article}
%%%\documentclass[12pt,a4paper]{scrartcl}
\usepackage[left=3cm,right=3cm,top=2cm,bottom=2cm]{geometry} % page settings
\usepackage{lingmacros}
\usepackage{tree-dvips}
\usepackage[polish]{babel}
%%% fix for \lll
\let\babellll\lll
\let\lll\relax 
\usepackage[T1]{fontenc}

\usepackage{amssymb}
\usepackage{verbatim}
\usepackage{listings}
\usepackage{amsmath}
\usepackage{amsthm}
\usepackage[boxed]{algorithm2e}


\title{Zadanie 2}
\author{Marko Golovko}
\date{\today}

\begin{document}
\maketitle
Wyznaczamy współczynnik korelacji między zachorowaniami. Dobranie odpowiedniego punktu początkowego, polega na wyznaczeniu dnia środkowego między pierwszym wypadkiem a zgonem. Dla każdego państwa liczymy korelację między zmiennymi określającymi wypadki a zgony. \\ 
Współczynnik korelacji ma wzór:
	$$ r = \frac{1}{n-1}\sum{(\frac{x_{i}-\overline{x}}{s_{x}})(\frac{y_{i}-\overline{y}}{s_{y}})} $$
Gdzie $n$ jest iliością obserwowanych dni. \\
$x_{i}$ ilość wypadków w i-ty dzień. $y_{i}$ ilość zgonów w i-ty dzień. \\
$$ \overline{x} = \frac{\sum{x_i}}{n} \quad \quad \overline{y} = \frac{\sum{y_i}}{n}$$
$$ s_{x} = \sqrt{\frac{\sum{(x_{i}-\overline{x})^2}}{n-1}} \quad \quad s_{y} = \sqrt{\frac{\sum{(y_{i}-\overline{y})^2}}{n-1}} $$
Wyniki obliczeń dołączam w oddzielnym pliku.
\cleardoublepage
\lstinputlisting[language=Python,
firstline=30, lastline=76]{z2.py}
\end{document}