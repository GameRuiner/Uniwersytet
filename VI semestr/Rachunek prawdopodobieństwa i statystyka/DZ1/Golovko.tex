
\documentclass[12pt]{article}
%%%\documentclass[12pt,a4paper]{scrartcl}
\usepackage[left=3cm,right=3cm,top=2cm,bottom=2cm]{geometry} % page settings
\usepackage{lingmacros}
\usepackage{tree-dvips}
\usepackage[polish]{babel}
%%% fix for \lll
\let\babellll\lll
\let\lll\relax 
\usepackage[T1]{fontenc}


\usepackage{amssymb}
\usepackage{verbatim}
\usepackage{listings}
\usepackage{amsmath}
\usepackage{amsthm}
\usepackage[boxed]{algorithm2e}

\renewcommand{\algorithmcfname}{Algorytm}
\SetKwInput{KwData}{\textbf{Dane}}
\SetKwInput{KwResult}{\textbf{Wynik}}
\renewcommand{\qedsymbol}{$\blacksquare$}

\title{Zadanie 1}
\author{Marko Golovko}
\date{\today}

\begin{document}
\maketitle

Standardowy rozkład normalny ma gęstość określoną wzorem
	$$ f(x) = \frac{1}{\sqrt{2\pi}}e^{-\frac{x^2}{2}}, x\in\mathbb{R}$$
Dla dystrybuanty otrzymujemy wyrażenie:
	$$ \Phi(t) = \int_{-\infty}^{t} \frac{1}{\sqrt{2\pi}}e^{-\frac{x^2}{2}} dx,$$
Ponieważ gęstość jest funkcją
parzystą zatem zachodzi związek $\Phi(t) = 1 - \Phi(-t)$.
\paragraph{Dowód:}
$$\Phi(t)= \int_{-\infty}^{t}f(x)dx = \int_{-\infty}^{\infty}f(x)dx - \int_{t}^{\infty}f(x)dx$$
Po pierwzse $ \int_{-\infty}^{\infty}f(x)dx = 1 $, oraz z tego że $f(x) = f(-x),$ wynika 
$$\int_{t}^{\infty}f(x)dx =  \int_{-\infty}^{-t}f(x)dx = \Phi(-t)$$
$$\Phi(t) = 1 - \Phi(-t)$$
\flushright\qedsymbol
\flushleft
$$ \Phi(t) = \int_{-\infty}^{t} \frac{1}{\sqrt{2\pi}}e^{-\frac{x^2} {2}} dx,$$
Rozbijemy całkę na dwie części.
$$ \Phi(t) = \frac{1}{\sqrt{2\pi}}(\int_{-\infty}^{0} e^{-\frac{x^2} {2}} dx + \int_{0}^{t} e^{-\frac{x^2} {2}} dx)$$
$$ \int_{-\infty}^{\infty} e^{-\frac{x^2} {2}} dx = \sqrt{2\pi} $$
\paragraph{Dowód:}
$$ \int_{-\infty}^{\infty} e^{-\frac{x^2} {2}} dx = 2\int_{0}^{\infty} e^{-\frac{x^2} {2}}dx = \sqrt{2}\int_{0}^{\infty} y^{\frac{-1}{2}}e^{-y}dy = \sqrt{2}\Gamma(1/2) $$
Warto zauważyć, że robimy zamianę $y = \frac{x^{2}}{2}$ zmiennych w tym dowodzie.\\
$\Gamma(1/2) = \sqrt{(\pi)} $ Wiemy to z ćwiczeń ( lista 5 zadanie 4).
$$ \int_{-\infty}^{\infty} e^{-\frac{x^2} {2}} dx = \sqrt{2\pi} $$
\flushright\qedsymbol
\flushleft
$$ \int_{-\infty}^{0} e^{-\frac{x^2} {2}} dx = \frac{\sqrt{2\pi}}{2} $$
Niech $ G(t) = \int_{0}^{t} e^{-\frac{x^2} {2}} dx $ wtedy 
$$ \Phi(t) = \frac{1}{\sqrt{2\pi}}(\frac{\sqrt{2\pi}}{2} + G(t)) $$
$$ \Phi(t) = \frac{1}{2} + \frac{1}{\sqrt{2\pi}}G(t) $$
Redukujemy zatem zadanie do postaci: \\
\quad Dla ustalonego t > 0 obliczyć wartość całki $$ G(t) = \int_{0}^{t} e^{-\frac{x^2} {2}} dx $$
Rozwiązanie opiera się na złożonym wzorze trapezów i metodzie Romberga.
\cleardoublepage
\lstinputlisting[language=Python,
firstline=1, lastline=38]{z5.jl}
\BlankLine
Dla $t=\{1,2,3\}$ mamy wartości $\Phi(t) = \{0.8413, 0.9772, 0.9987\}$ odpowiednio.

\end{document}