
\documentclass[12pt]{article}
%%%\documentclass[12pt,a4paper]{scrartcl}
\usepackage[left=3cm,right=3cm,top=2cm,bottom=2cm]{geometry} % page settings
\usepackage{lingmacros}
\usepackage{tree-dvips}
\usepackage{latexsym}
\usepackage[polish]{babel}

%\usepackage{vhistory}
\usepackage[owncaptions]{vhistory}


\usepackage[T1]{fontenc}

\usepackage{epigraph}
\setlength{\parskip}{1em}
\title{SpeedCheck. Specyfikacja wymagań}
\author{Marko Golovko. Yurii Purdenko}
\date{\today}



\begin{document}
\begin{titlepage}
\maketitle
\end{titlepage}

\renewcommand{\vhhistoryname}{Historia zmian}
\newcommand{\MG}{Marko Golovko}
\begin{versionhistory}
  \renewcommand{\vhhistoryname}{Histria zmian}
  \renewcommand{\vhversionname}{Numer wersji}  
  \renewcommand{\vhdatename}{Data} 
  \renewcommand{\vhauthorname}{Autor} 
  \renewcommand{\vhchangename}{Opis}
  \vhEntry{0.1}{08.11.20}{MG}{Utworzenie dokumentu}
  \vhEntry{0.2}{08.11.04}{MG}{Określenie wymagań}
  \vhEntry{0.3}{13.11.20}{MG}{Korekta dokumentu}
\end{versionhistory}

\tableofcontents

\section{Określenie wymagań}
\subsection{Historyjki użytkownika}
\begin{itemize}
\item Jako użytkownik chce wyszukiwać samochód za numerem, aby ułatwić znalezienie skradzionego samochodu.
\item Jako użytkownik (np. pracownik służby kontroli ruchu) chce mieć dostęp do oglądania transmisji z kamer, aby natychmiastowo reagować na wypadki drogowe.
\item Jako użytkownik (np. pracownik służby kontroli ruchu) chce widzieć prędkość samochodów, aby mieć możliwość uprzedzić kierowcę o przekroczeniu limitu prędkości.
\end{itemize}
\subsection{Główne systemy składowe}
\subsubsection*{Wyszukiwanie samochodów}
System wyszukiwania zaciekawi głównie użytkowników będących pracownikami policji. Po nadaniu danych o skradzionym samochodzie system wyszukuje trafienia w nagraniach i na żywo. Precyzyjne wyszukiwanie dostępne za numerem samochodu, ale nie zawsze  system może zdefiniować numer samochodu z nagrania, więc dostępne dodatkowe parametry dla wyszukiwania np. lokalizacja, czas lub kolor samochodu. 
\subsubsection*{Oglądanie transmisji z kamer}
System oglądania transmisji daje możliwość użytkowniku oglądać sytuację na drogach na żywo. Użytkownik może wybrać lokalizację, która go interesuje. Razem s transmisją użytkownik otrzymuję statu stanu drogi i limit prędkości na danym odcinku.  Jest możliwość multitransmicji. 
\subsubsection*{Oglądanie nagrań z bazy danych}
Nagranie z kamer przechowywane przez 30 dni. Użytkownik ma możliwość wybrania lokalizacji i czasu nagrania.
\subsection{Wymagania funkcjonalne}
\begin{itemize}
\item Wyszukiwanie samochodu. Użytkownik wprowadza potrzebne dane, system wypisuję listę trafień.
\item Oglądanie kamer na żywo. Użytkownik wprowadza potrzebne dane, system wypisuje listę dostępnych transmisji.
\item Oglądanie nagrań. Użytkownik wprowadza potrzebne dane, system wypisuje listę nagrań zgodnie z danymi użytkownika.
\item Informowanie o przekroczeniu prędkości. Użytkownik wybiera interesujące go translacji i ustawia system na alarmowanie przy przekroczeniu prędkości.
\item Dodawanie transmisji do systemu. Administrator wpisuje potrzebne dany i ustawia transmisję.
\item Oglądanie multitransmicji. Użytkownik może oglądać do 9 transmisji jednocześnie..
\item Możliwość rejestracji użytkowników. Profil użytkownika ma dane o trafieniach na kamerach i historię oglądanych transmisji i nagrań. 
\item Możliwość pobrania fragmentu nagrania. Użytkownik może pobrać fragment nagrania w formacie mp4.
\item Możliwość wbudowania transmisji na stronę użytkownika. Za pomocą wygenerowanego fragmentu kodu użytkownik pobiera informacje o transmisji dla wbudowania na swojej stronie.
\item Możliwość proponowania transmisji przez użytkowników. Użytkownik może dodać dane do transmisji, i administrator rozpatrzy propozycję.
\end{itemize}
\subsection{Wymagania niefunkcjonalne}
\begin{itemize}
\item Szybkość aplikacji. Sprawdzanie liczby transakcji na sekundę.
\item Wykorzystanie pamięci operacyjnej. Sprawdzanie zajętości w procentach pamięci RAM	
\item Łatwość użycia. Test UX.
\item Niezawodność aplikacji. Sprawdzanie liczby błędów na liczbę godzin użytkowania.
\item Działanie aplikacji przy dużej liczbie użytkowników. Stress test.
\end{itemize}
\section*{Przypadki użycia}

\end{document}