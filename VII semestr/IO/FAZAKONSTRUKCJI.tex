
\documentclass[12pt]{article}
%%%\documentclass[12pt,a4paper]{scrartcl}
\usepackage[left=3cm,right=3cm,top=2cm,bottom=2cm]{geometry} % page settings
\usepackage{lingmacros}
\usepackage{tree-dvips}
\usepackage{latexsym}
\usepackage[polish]{babel}
\usepackage{polski}
\usepackage[utf8]{inputenc}

\usepackage[pdftex]{graphicx}
%\usepackage{vhistory}
\usepackage[owncaptions]{vhistory}


\usepackage[T1]{fontenc}

\usepackage{epigraph}
\setlength{\parskip}{1em}
\title{SpeedCheck. Faza konstrukcji}
\author{Marko Golovko. Yurii Purdenko}
\date{\today}



\begin{document}
\begin{titlepage}
\maketitle
\end{titlepage}

\renewcommand{\vhhistoryname}{Historia zmian}
\newcommand{\MG}{Marko Golovko}
\begin{versionhistory}
  \renewcommand{\vhhistoryname}{Histria zmian}
  \renewcommand{\vhversionname}{Numer wersji}  
  \renewcommand{\vhdatename}{Data} 
  \renewcommand{\vhauthorname}{Autor} 
  \renewcommand{\vhchangename}{Opis}
  \vhEntry{0.1}{14.12.20}{YP}{Utworzenie dokumentu}
 \end{versionhistory}

\tableofcontents

\section{Testy funkcjonalne}

\subsection{Wyszukiwanie samochodów}

\subsubsection{User Story}
\begin{itemize}
\item Nazwa: Wyszukiwanie samochodu
\item Historyjka użytkownika: Jako użytkownik, chce mieć możliwość wyszukiwania samochodu, aby widzieć go ostatnią lokalizację.
\item Funkcjonalne kryteria akceptacji: 
	\begin{itemize}
	\item Strona wyświetla lokalizacje samochodu.
	\item Jeżeli lokalizacja nie była znaleziona, wyświetla komunikat, że nie udało się znaleźć.
	\item Jeżeli użytkownik nie ma prawa do wyszukiwania,  wyświetla komunikat, że nie ma prawa do wyszukiwania.
	\end{itemize}
\end{itemize}

\subsubsection{Use case}
\begin{itemize}
\item Tytul: Wyszukiwanie samochodu
\item Aktor: Użytkownik
\item Warunki początkowe: 
	\begin{itemize}
	\item Użytkownik muśi być zalogowany.
	\item Użytkownik muśi mieć prawo do wyszukiwania samochodu.
	\item Strona musi mieć pole do wyszukiwania.
	\end{itemize}
\item Zdarzenie inicujące: Użytkownik wpisuje numer samochodu w odpowiednie pole i klika przycisk wyszukiwania.
\item Przebieg w krokach:
	\begin{itemize}
	\item Użytkownik loguje się na stronie.
	\item Użytkownik wpisuje numer samochodu w odpowiednie pole i klika przycisk wyszukiwania
	\item Strona wyświetla lokalizacje samochodu.
	\end{itemize}
\item Przebiegi alterantywne: 
	\begin{itemize}
	\item Wyszukiwany samochód nie był znaleziony.
	\item Użytkownik nie ma prawa do wyszukiwania samochodu.
	\end{itemize}
\item Sytuacji wyjątkowe: Użytkownik jest pracownikiem policji i ma prawo do wyszukiwania dowolnego samochodu.
\item Warunki końcowe:
	\begin{itemize}
	\item Strona wyświetla lokalizacje samochodu.
	\item Albo strona wyświetla komunikat o nieudanej próbie.
	\end{itemize}
\end{itemize}

\subsection{Przegląd kamer monitorujących}

\subsubsection{User Story}
\begin{itemize}
\item Nazwa: Przegląd kamer monitorujących
\item Historyjka użytkownika: Jako użytkownik, chce mieć możliwość przeglądu kamer monitorujących, by wiedzieć o sytuacji na drogach
\item Funkcjonalne kryteria akceptacji: 
	\begin{itemize}
	\item Strona wyświetla dostępne transmisji.
	\item Jest możliwość filtracji za lokalizacją.
	\item Jest możliwość oglądania kilku kamer(do 9) jednocześnie.
	\end{itemize}
\end{itemize}

\subsubsection{Use case}
\begin{itemize}
\item Tytul: Przegląd kamer monitorujących
\item Aktor: Użytkownik
\item Warunki początkowe: 
	\begin{itemize}
	\item Użytkownik muśi być zalogowany.
	\item Strona ma dostępne transmisje
	\item Użytkownik znajduje się w sekcji „Przegląd kamer online”
	\end{itemize}
\item Zdarzenie inicujące: Użytkownik klika na zakładkę „Przegląd kamer online”
\item Przebieg w krokach:
	\begin{itemize}
	\item Użytkownik loguje się na stronie.
	\item Użytkownik zaznacza lokalizacje kamer z transmisją.
	\item Strona wyświetla transmisji.
	\end{itemize}
\item Przebiegi alterantywne: 
	\begin{itemize}
	\item Transmisji nie były znalezione.
	\item Użytkownik dodaje kolejną transmisję do oglądania.
	\end{itemize}
\item Sytuacji wyjątkowe: Użytkownik chcę dodać więcej niż dziewięć transmisji na raz.
\item Warunki końcowe:
	\begin{itemize}
	\item Strona wyświetla dostępne transmisji.
	\item Albo strona wyświetla komunikat o nieudanej próbie.
	\end{itemize}
\end{itemize}

\subsection{Sprawdzenia przekroczeń limitu prędkości}

\subsubsection{User Story}
\begin{itemize}
\item Nazwa: Sprawdzenia przekroczeń limitu prędkości
\item Historyjka użytkownika: Jako użytkownik, chce mieć możliwość sprawdzenia przekroczeń limitu prędkości, by mieć możliwość zabezpieczyć bezpieczną sytuację na drogach.
\item Funkcjonalne kryteria akceptacji: 
	\begin{itemize}
	\item Strona wyświetla dostępne transmisji.
	\item Jest możliwość filtracji za lokalizacją.
	\item Jest możliwość oglądania kilku kamer(do 9) jednocześnie.
	\item Na transmisji jest pokazane przekroczenie prędkości.
	\end{itemize}
\end{itemize}

\subsubsection{Use case}
\begin{itemize}
\item Tytul: Sprawdzenia przekroczeń limitu prędkości
\item Aktor: Użytkownik
\item Warunki początkowe: 
	\begin{itemize}
	\item Użytkownik muśi być zalogowany.
	\item Musi mieć prawo do sprawdzenia prędkości wszystkich samochodów (np. być pracownikiem policji).
	\item Strona ma dostępne transmisje.
	\item Użytkownik znajduje się w sekcji „Sprawdzenia przekroczeń limitu”.
	\end{itemize}
\item Zdarzenie inicujące: Użytkownik klika na zakładkę „Sprawdzenia przekroczeń limitu”
\item Przebieg w krokach:
	\begin{itemize}
	\item Użytkownik loguje się na stronie.
	\item Użytkownik zaznacza lokalizacje kamer z transmisją.
	\item Strona wyświetla transmisji i pokazuje przekroczenia prędkości.
	\end{itemize}
\item Przebiegi alterantywne: 
	\begin{itemize}
	\item Transmisji nie były znalezione.
	\item Użytkownik dodaje kolejną transmisję do oglądania.
	\end{itemize}
\item Sytuacji wyjątkowe: użytkownik nie jest pracownikiem policji i ma możliwość do sprawdzenia prędkości tylko własnego samochodu.
\item Warunki końcowe:
	\begin{itemize}
	\item Strona wyświetla dostępne transmisji.
	\item Strona pokazuje przekroczenia prędkości.
	\item Albo strona wyświetla komunikat o nieudanej próbie.
	\end{itemize}
\end{itemize}

\section{Pomiary spełnienia wymagań niefunkcjonalnych}
    \subsection{Wymaganie niefunkcjonalne}
    \begin{itemize}
        \item \subsection{Szybkość aplikacji. aplikacja musi być szybka}
              \begin{itemize}
                    \item Pomiar: X = A/N 
                    \item A = 
              
              \end{itemize}
        \item Wykorzystanie pamięci operacyjnej. Sprawdzanie zajętości w procentach pamięci RAM	
        \item Łatwość użycia. Test UX.
        \item Niezawodność aplikacji. Sprawdzanie liczby błędów na liczbę godzin użytkowania.
        \item Działanie aplikacji przy dużej liczbie użytkowników. Stress test.
    \end{itemize}
\section{Plan beta testowania}
\subsection{Planowanie}
Zdefiniowanie takich elementów:
\begin{itemize}
\item Zakres testów i cel testów
\item Elementu testowego
\item Funkcjonalności, które zostaną przetestowane
\item Poziomy i typy testów
\item Kryteria akceptacji
\item Rzeczy które są dostarczane razem z testami
\item Środowisko testowe
\item Harmonogram
\item Skład teamu 
\item Ryzyka i zagrożenia
\end{itemize}

\subsection{Analiza}
\begin{itemize}
\item Określamy przedmiot testowy
\item Określamy cel testowania
\item Robimy spis funkcjonalności i możliwości, które będą testowane.
\item Z otrzymanej informacji definiujemy ankiety testowe.
\item Określamy elementy testowe
\item Dokonujemy analizy ryzyka
\end{itemize}

\subsection{Projektowanie testów}
\begin{itemize}
\item Wyznaczamy poziomy testów dla konkretnych testowanych obszarów
\item Wybieramy techniki projektowania 
\item Tworzymy ankiety testowe
\item Priorytetyzacja ankiet testowych
\end{itemize}

\subsection{Implementacja}
\begin{itemize}
\item Przygotowujemy środowisko testowe 
\item Testują ochotnicy manualne
\item Team testerów kieruje proces beta testowania
\end{itemize}

\subsection{Wykonanie testów}
\begin{itemize}
\item Użytkownicy wykonują testy na naszym elemencie testowym
\item Po wykonaniu testów otrzymane wrażenie zapisują w zawartości aktualnej ankiety
\item Ocenę użytkowników team testerów przetwarza w raport
\item Zgodnie z raportem część przetestowanej funkcjonalności przechodzi z powrotem do fazy projektowania
\item Zgodnie z uwzględnieniem pewnych odstępstw, plan może ulecz zmianie.
\end{itemize}

\subsection{Kontrola}
\begin{itemize}
\item Monitorowanie
\item Raportowanie
\item Nadzór
\end{itemize}

\subsection{Czynności zamykające testowanie}
\begin{itemize}
\item Sprawdzić, czy wszystko zostało dostarczone zgodnie z planem i dokumentacją
\item Zarządzić błędami, czyli zamykanie notek, tworzenie nowych, edycja istniejących, jeśli jest taka potrzeba
\item Utrzymywanie dokumentacji
\item Prace niezbędne do utrzymania środowiska testowego i ponownego użycia w przyszłości
\item Zorganizować spotkanie retrospektywne
\item Zachować artefakty procesu testowego
\end{itemize}

\section{Plan zarządzania ryzykiem}


\section{Plan zarządzania jakością wytwarzania oprogramowania}


\section{Ocena zgodności wykonanych prac z wizją systemu i specyfikacją wymagań.}

Rezultaty są zgodne z wizją systemu i specyfikacją wymagań.

\end{document}