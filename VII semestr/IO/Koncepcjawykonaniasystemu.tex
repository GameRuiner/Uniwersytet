
\documentclass[12pt]{article}
%%%\documentclass[12pt,a4paper]{scrartcl}
\usepackage[left=3cm,right=3cm,top=2cm,bottom=2cm]{geometry} % page settings
\usepackage{lingmacros}
\usepackage{tree-dvips}
\usepackage{latexsym}
\usepackage[polish]{babel}
\usepackage{polski}
\usepackage[utf8]{inputenc}
\usepackage{graphicx}

%\usepackage{vhistory}
\usepackage[owncaptions]{vhistory}


\usepackage[T1]{fontenc}

\usepackage{epigraph}
\setlength{\parskip}{1em}
\title{SpeedCheck. Koncepcja wykonania systemu}
\author{Marko Golovko. Yurii Purdenko}
\date{\today}



\begin{document}
\begin{titlepage}
\maketitle
\end{titlepage}

\renewcommand{\vhhistoryname}{Historia zmian}
\newcommand{\MG}{Marko Golovko}
\begin{versionhistory}
  \renewcommand{\vhhistoryname}{Histria zmian}
  \renewcommand{\vhversionname}{Numer wersji}  
  \renewcommand{\vhdatename}{Data} 
  \renewcommand{\vhauthorname}{Autor} 
  \renewcommand{\vhchangename}{Opis}
  \vhEntry{0.1}{21.11.20}{MG}{Utworzenie dokumentu}
  \vhEntry{0.2}{21.11.20}{YP}{Przypadki użycia}
  \vhEntry{0.3}{21.11.20}{MG}{UI}
  \vhEntry{0.4}{23.11.20}{YP}{Schemat bazy danych}
  \vhEntry{0.5}{23.11.20}{MG}{Ryzyka}
  \vhEntry{0.6}{23.11.20}{YP}{Ocena zgodności z tablicą koncepcyjną}
  \vhEntry{0.7}{24.11.20}{MG}{Model konceptualny}
  \vhEntry{1.0}{24.11.20}{YP}{Architektura}
\end{versionhistory}

\tableofcontents

\section{Przypadki użycia}
	
	\subsection{Logowanie/Rejestracja}
	\begin{itemize}
	\item klient wchodzi na stronę systemu SpeedCheck
	\item klient widzi stronę Logowania/Rejestracji
	\item klient jest zarejestrowany, wprowadza dane do logowania
	\item klient nie jest zarejestrowany, przechodzi do strony rejestracji i wprowadza wymagane dane oraz czeka na weryfikację danych (imie, nazwisko, e-mail, numer telefonu, dolącza dokument tózsamości oraz dokument na swój samochód)
	\item otwiera się glówna strona systemu SpeedCheck 
	\end{itemize}
	
	\subsection{Wyszukiwanie samochodów}
	\begin{itemize}
	\item klient jest poprawnie zalogowany na stronie systemu SpeedCheck
	\item klient chce wiedzieć, gdzie samochód zostal ostatnie widziany  
	\item klient klika na podzakładkę “Wyszukać samochód”
	\item system sprawdza, czy klient ma prawo do wyszukiwania innych samochodów niż jego własny(np. kilent jest pracownikem policji), jeśli tak - otworzy się strona wprowadzania danych, jeśli nie - system automatycznie wprowadza dane samochoda klienta.
	\item otwiera się strona z danymi, na których kamerach i kiedy samochód zostal ostatnio oglądany 			widziany
	\end{itemize}
	
	\subsection{Przegląd kamer monitorujących}
	\begin{itemize}
	\item klient jest poprawnie zalogowany na stronie systemu SpeedCheck
	\item klient chce spojrzeć na kamery monitorujące online
	\item klient klika na podzakładkę “Przegląd kamer online”
	\item otworzy się strona wprowadzania danych kamery, którą klient chce spojrzeć
	\item po wprowadzeniu danych system utworzy transmisję
	\end{itemize}
	
	\subsection{Ustawienia limitu prędkości na kamerach}
	\begin{itemize}
	\item klient jest poprawnie zalogowany na stronie systemu SpeedCheck
	\item klient chce ustawić limit prędkości na kamerach
	\item klient klika na podzakładkę “Ustawienia limitu prędkości na kamerach”
	\item otworzy się strona wprowadzania danych kamery oraz limitu prędkości (jeśli klient posiada na to prawo np. jest pracownikiem kontrolu ruchu)
	\item po wprowadzeniu danych system ustawia limit oraz zapisuje do bazy danych numery samochodów, które
	przekręcili limit prędkości	
	\end{itemize}
	
	\subsection{Sprawdzenia wszystkich przekroczeń limitu prędkości samochodem}
	\begin{itemize}
	\item klient jest poprawnie zalogowany na stronie systemu SpeedCheck
	\item klient chce sprawdzić wszystkie przekroczenia limitu prędkości samochodem
	\item klient klika na podzakładkę “Sprawdzenia przekroczeń limitu”
	\item system sprawdza, czy klient ma prawo do sprawdzenia innych samochodów niż jego własny(np. kilent 	jest pracownikem policji), jeśli tak - otworzy się strona wprowadzania danych, jeśli nie - system automatycznie wprowadza dane samochoda klienta.
	\item otwiera się strona z danymi, gdzie i kiedy samochód przekręcil limit prędkości
	\end{itemize}
	
\section{UI}
Patrz katalog UI.

\section{Architektura}

\subsection{Sprzęt}
	Jako server będziemy wykorzystywać AWS Elastic Compute Cloud (Amazon EC2) z 			wykorzystaniem Docker oraz AWS CloudFront(CDN). Wszystkie resursy bedą opisane w AWS 			CloudFormation w formacie yaml.
    
\subsection{Oprogramowanie}
\begin{itemize}
	\subsubsection{Narzędzia programistyczne}
    	\item  Będziemy wykorzystywać Visual Studio Code z wtyczkami ułatwiającymi pracę nad
		częścią kliencką, Yarn jako menadżer pakietów oraz przeglądarki Google Chrome i Mozilla
		Firefox.
        
	\subsubsection{Bazy danych.}
    	\item Jako Bazę Danych będziemy wykorzystować PostgreSQL.
	\subsubsection {Oprogramowanie do automatycznego testowania.}
    	\item Robot Framework z z biblioteką SeleniumLibrary.
\end{itemize}


\subsection{Model konceptualny}

\includegraphics{model.png}


\subsection{Schemat bazy danych}
	%Patrz database.png
\includegraphics{database.png}
	
\section{Główne zasady kodowania}
\begin{itemize}
	\item Google JavaScript Style Guide
	\item PEP 8. Style Guide for Python Code
\end{itemize}


\section{Identyfikacja ryzyka}

\subsection{Identyfikacja}
\begin{itemize}
	\item Ryzyko techniczne. Na przykład nie są pewni, czy dane wymaganie jest osiągalne, biorąc pod uwagę ograniczenia wynikające z istniejącej technologii.
	\item Łańcuch dostaw. Na przykład możemy mieć problem z łatwym pozyskaniem sprzętu wysokiej jakości
	\item Ryzyka wytworzenia. Czy jesteśmy w stanie wytworzyć kod szybko i wystarczająco niezawodnie, aby spełnić harmonogram i cele dotyczące wydatków projektu?
	\item Zarządzanie programem. Duża zmiana zakresu w trakcie projektu.
	\item Interpersonalne. Na przykład, jeśli brakuje skutecznej komunikacji między członkami zespołu programistycznego lub między zespołem a kierownictwem.
	\item Nieznane zagrożenia. Nawet mając najlepsze planowanie i najlepsze intencje, nie da się znaleźć wszystkiego.
\end{itemize}

\subsection{Ocena ryzyka}
\begin{itemize}
\item Ryzyko techniczne. Dla zmniejszenia ryzyka wystąpienia zdarzenia możemy zrobić analizę podobnych projektów. Pierwszy raz robimy podobny projekt, więc prawdopodobieństwa wystąpienia jest wysokie.
\item Łańcuch dostaw. Nie potrzebujemy unikatowego sprzętu dla pracy nad projektem, więc wystąpienie problem z tym zdarzeniem jest minimalne.
\item Zarządzanie programem. Poświecimy dużo czasu na określenie specyfikacji projektu, aby zminimalizować wystąpienie tego zdarzenia.
\item Interpersonalne. Tryb pracy zdalnej może mieć krytyczny wpływ na komunikację w zespole. 
\item Nieznane zagrożenia. Wysokie prawdopodobieństwo wystąpienia niewiadomych zagrożeń.
\end{itemize}

\subsection{Planowanie}
Z oceny ryzyka widzimy wysokie szansy na wystąpienia niekorzystnych dla projektu zdarzeń. Dla zapobiegania nim planujemy skorzystać z konsultacji menedżerów podobnych projektów. 

\section{Ocena zgodności z tablicą koncepcyjną}
	Ogólnie rezultaty są zgodne z tym co zostało przedstawione w tablicy
	koncepcyjnej. Jedyną różnicą jest czas przeznaczony na testowanie produktu.
	Stwierdziliśmy że zamiast 2 miesięcy, przeznaczymy na to 3 miesięcy

\end{document}