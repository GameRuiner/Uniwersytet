
\documentclass[12pt]{article}
%%%\documentclass[12pt,a4paper]{scrartcl}
\usepackage[left=3cm,right=3cm,top=2cm,bottom=2cm]{geometry} % page settings
\usepackage{lingmacros}
\usepackage{tree-dvips}
\usepackage{latexsym}
\usepackage[polish]{babel}
\usepackage{polski}
\usepackage[utf8]{inputenc}

\usepackage[pdftex]{graphicx}
%\usepackage{vhistory}
\usepackage[owncaptions]{vhistory}


\usepackage[T1]{fontenc}

\usepackage{epigraph}
\setlength{\parskip}{1em}
\title{SpeedCheck. Faza wprowadzenie na rynek}
\author{Marko Golovko. Yurii Purdenko}
\date{\today}



\begin{document}
\begin{titlepage}
\maketitle
\end{titlepage}

\renewcommand{\vhhistoryname}{Historia zmian}
\newcommand{\MG}{Marko Golovko}
\begin{versionhistory}
  \renewcommand{\vhhistoryname}{Histria zmian}
  \renewcommand{\vhversionname}{Numer wersji}  
  \renewcommand{\vhdatename}{Data} 
  \renewcommand{\vhauthorname}{Autor} 
  \renewcommand{\vhchangename}{Opis}
  \vhEntry{0.1}{30.01.21}{MG}{Utworzenie dokumentu}
  \vhEntry{0.2}{30.01.21}{MG}{Oszacowanie pojemności bazy danych aplikacji}
  \vhEntry{0.3}{30.01.21}{YP}{Koncepcja organizacji szkoleń użytkowników}
  \vhEntry{0.4}{31.01.21}{YP}{Sformułowanie głównych punktów umów}
  \vhEntry{0.5}{1.02.21} {MG} {Plan wdrożenia}
  \vhEntry{0.6}{1.02.21} {YP} {Koncepcja wsparcia technicznego}
  \vhEntry{1}  {1.02.21} {MG} {Sposób pomiaru satysfakcji klienta}
 \end{versionhistory}

\tableofcontents

\section{Oszacowanie pojemności bazy danych aplikacji}
Baza danych ma następujące tabele:
\begin{itemize}
\item \textbf{users}: tabela zawierająca dane użytkowników 
	\begin{itemize}
		\item \textbf{Wielkość wiersza (bajty):} 100
		\item \textbf{Wiersze na początku eksploatacji:} 0
		\item \textbf{Łączna wielkość:} 0 MB
		\item \textbf{Oszacowanie miesięcznego przyrostu (wierszy):} 1000
		\item \textbf{Oszacowanie miesięcznego przyrostu:} <1 MB
	\end{itemize}
\item \textbf{carsLocation}: tabela zawierająca miejsca znalezienia samochodów
	\begin{itemize}
		\item \textbf{Wielkość wiersza (bajty):} 8
		\item \textbf{Wiersze na początku eksploatacji:} 0
		\item \textbf{Łączna wielkość:} 0 MB
		\item \textbf{Oszacowanie miesięcznego przyrostu (wierszy):} ilość godzin działania kamer*250
		\item \textbf{Oszacowanie miesięcznego przyrostu:} przyrost wierszy * 8 
	\end{itemize}
\item \textbf{speeding}: tabela zawierająca dane o przekroczeniu prędkości 
	\begin{itemize}
		\item \textbf{Wielkość wiersza (bajty):} 12
		\item \textbf{Wiersze na początku eksploatacji:} 0
		\item \textbf{Łączna wielkość:} 0 MB
		\item \textbf{Oszacowanie miesięcznego przyrostu (wierszy):} ilość godzin działania kamer*10
		\item \textbf{Oszacowanie miesięcznego przyrostu:} przyrost wierszy * 12 
	\end{itemize}
\item \textbf{car}: tabela zawierająca dane o samochodach
	\begin{itemize}
		\item \textbf{Wielkość wiersza (bajty):} 40
		\item \textbf{Wiersze na początku eksploatacji:} 0
		\item \textbf{Łączna wielkość:} 0 MB
		\item \textbf{Oszacowanie miesięcznego przyrostu (wierszy):} ilość godzin działania kamer*250 - powtarzające się samochody
		\item \textbf{Oszacowanie miesięcznego przyrostu:} przyrost wierszy * 40 
	\end{itemize}
\item \textbf{location}: tabela zawierająca dane lokalizacji kamer
	\begin{itemize}
		\item \textbf{Wielkość wiersza (bajty):} 50
		\item \textbf{Wiersze na początku eksploatacji:} 10
		\item \textbf{Łączna wielkość:} < 1 MB
		\item \textbf{Oszacowanie miesięcznego przyrostu (wierszy):} 50
		\item \textbf{Oszacowanie miesięcznego przyrostu:} < 1 MB
	\end{itemize}
\item \textbf{cameras}: tabela zawierająca dane kamer
	\begin{itemize}
		\item \textbf{Wielkość wiersza (bajty):} 100
		\item \textbf{Wiersze na początku eksploatacji:} 10
		\item \textbf{Łączna wielkość:} < 1 MB
		\item \textbf{Oszacowanie miesięcznego przyrostu (wierszy):} 50
		\item \textbf{Oszacowanie miesięcznego przyrostu:} 60 GB zawartości kamery * 50
	\end{itemize}
\item \textbf{permisions}: tabela zawierająca role użytkowników. Rozmiar wiersza 40 bajtów.  Łączna wielkość i przyrost jest mniejszy od 1 MB. 
\end{itemize}

Informacja niezbędna do rozpoczęcia eksploatacji aplikacji:
\begin{itemize}
\item tabela \textbf{permisions}
\item 10 pracujących kamer z danymi w tabelach \textbf{cameras} i \textbf{location}
\end{itemize}


\section{Plan wdrożenia}

\begin{itemize}
	\item Przygotowanie dokumentacji powykonawczej (zawiera szczegółowy opis wykonanych czynności instalacyjnych oraz konfiguracyjnych
wszystkich komponentów systemu)
	\item Przygotowanie i skonfigurowanie infrastruktury technicznej (konfigurowanie kamer i transmisji)
	\item Zainstalowanie i skonfigurowanie systemu informatycznego do testów
	\item Testowanie systemu
	\item Testowa migracja danych
	\item Zainstalowanie i skonfigurowanie systemu informatycznego do eksploatacji
	\item Uruchomienie produkcyjne systemu - rozpoczęcie pracy przez użytkowników
\end{itemize}


\section{Koncepcja organizacji szkoleń użytkowników}
	\begin{itemize}
    \item Stworzenie dokumentu opisującego stronę, jej funkcjonalności, treści w niej
	zawartych. Dokument ten można zamieścić na stronie głównej serwisu (do użytku
	użytkowników, jako krótki poradnik).
    \item Szkolenie funkcjonariuszy kontroli ruchu i policji w zakresie wszystkich funkcji systemu (jak szukać samochodu, jak analizować odpowiedź z bazy danych, jak przeglądać kamery itp.)
	\item Szkolenie zespołu obsługi klienta (szkolenie poprawnej rozmowy z klientem, umiejętności odpowiadania na pytania dotyczące systemu itp.)
    \item Szkolenie administratorów serwera
	\item Szkolenie administratorów systemu w zakresie sprawdzania dokumentów i dodawania nowych 
	klientów do bazy danych
    \end{itemize}
Po szkoleniu i kilku miesiącach pracy otrzymamy 'feedback' od klientów i poprawimy wszystkie błędy.


\section{Koncepcja wsparcia technicznego}
Aby uzyskać pomoc techniczną, będzie dostępna sekcja „FAQ”, dostępna również pomoc przez czat dostępny na stronie systemu.
Będzie też zespół administratorów systemu i programistów do rozwiązywania problemów technicznych zglaszanych przez klientów.

\section{Sformułowanie głównych punktów umów}

Wprowadzimy kompleks umów:
	\begin{itemize}
 	\item Umowę o analizę przedwdrożeniową
 	\item Umowę wdrożeniową
	\item Umowę prawnoautorską
 	\item Umowę serwisową
 	\item Umowę o zakresie usług serwisowych( Usuwanie błędów, wsparcie techniczne itp.)
 	\end{itemize}
 Także użyjemy pewnej formy gwarancji należytego wykonania umowy  dla gwarancji  płynności finansowej w postaci:
 \begin{itemize}
 \item Gwarancji bankowej
 \item Gwarancje ubezpieczeniowej
 \end{itemize}
Szczególną uwagę przydzielimy opisowi kryteriów, procedury i metod odbioru produktu IT.


\section{Sposób pomiaru satysfakcji klienta}

\subsection*{Współczynnik retencji klienta}
Dla stałych użytkowników (np. pracowników policji) będziemy stosować współczynnik retencji klienta.
Wzór na współczynnik retencji klienta = ((E-N) / S) x 100
Do obliczenia współczynniku retencji klienta, potrzebujemy trzech informacji:
\begin{itemize}
	\item Liczba nowych klientów w danym okresie (N)
	\item Liczba klientów na koniec danego okresu (E)
	\item Liczba klientów na początku danego okresu (S)
\end{itemize}

\subsection*{Rozwiązanie sprawy klienta w pierwszym kontakcie}
Dla użytkowników (np. pracowników policji) będziemy stosować współczynnik rozwiązanie sprawy klienta w pierwszym kontakcie.
Mierzymy obliczając stosunek zgłoszeń klientów rozwiązanych „od ręki” – w pierwszym kontakcie – do wszystkich zgłoszeń i mnożony przez 100%

\end{document}